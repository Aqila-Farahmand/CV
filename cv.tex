\documentclass{resume} % Use the custom resume.cls style
\usepackage{cv}
\PassOptionsToPackage{hyphens}{url}
\usepackage{hyperref}
%\addbibresource{bibliography.bib}

%----------------------------------------------------------------------------------------
%	PERSONAL DATA
%----------------------------------------------------------------------------------------\\

\name{Aqila Farahmand}
\phone{(+39) 351 671 5137}
\birthday{October 4, 1995}
\genre{Female}
\country{Italy}
\town{47521, Cesena (FC)}
%\address{via dell'Università 50}
%\email{\href{mailto:matteo.magnini@studio.unibo.it}{\texttt{matteo.magnini@studio.unibo.it}}}
\email{\href{mailto:aqela.af@gmail.com}{\texttt{aqela.af@gmail.com}}}
\github{\href{https://github.com/Aqila-Farahmand}{github.com/Aqila-Farahmand}}
\linkedin{\href{https://www.linkedin.com/in/aqila-farahmand/}{linkedin.com/in/aqila-farahmand}}
%\orcid{\href{https://orcid.org/0000-0001-9990-420X}{\texttt{0000-0001-9990-420X}}}
%\googlescholar{\href{https://scholar.google.com/citations?user=iJDPEaUAAAAJ\&hl=en}{\texttt{iJDPEaUAAAAJ}}}
\photo{aqila.jpg}

\begin{document}
    
    \begin{rSection}{In short}
        I recently earned my Master’s Degree in Computer Science from the University of Camerino, where I completed a research thesis at the University of Oslo.
        %
        Currently, I am working as an intern with \href{https://www.sintef.no/}{SINTEF} on the \href{https://www.themis-trust.eu/}{THEMIS.5} project.
    \end{rSection}
    
    %----------------------------------------------------------------------------------------
    %	EDUCATION AND CARREER SECTION
    %----------------------------------------------------------------------------------------
    \begin{rSection}{Education and Carreer}

        %----------------------------------------------------------------------------------------
        %	EDUCATION SECTION
        %----------------------------------------------------------------------------------------
        
        \begin{rSubsection2}{Education}

            \item\textbf{ Master's degree }\hfill \textbf{Mar., 2022 $\rightarrow$ Oct., 2024}
            \\Computer Science (Intelligent and Adoptive Systems), University of Camerino, Italy.
            \\Studies in the field of programming paradigms, complex systems, distributed systems, machine learning, web applications, data mining.
            \\106/110
            %
            \item\textbf{ Master's degree thesis }\hfill \textbf{Aug., 2023 $\rightarrow$ Jul., 2024}
            \\Identification of Workflows and Files contributing to TD through Commit-Level Analysis across the GitHub Action
            \\Relator: Prof. Flavio Corradini
            \\Co-relator: Prof. Antonio Martini
            \\We developed a model to mine data from GitHub Actions and employed an NLP model for automatic text classification.
            Our model identifies files and workflows within GitHub Actions that are likely to introduce technical debt (TD) and visualizes them.
            This represents a significant advancement in TD identification and mitigation during software development processes.
            %
            \item\textbf{ Bachelor of Technology }\hfill \textbf{Jul., 2016 $\rightarrow$ Jul., 2021}
            \\Electronics and Communication Engineering, NIT Kurukshetra, India (standard duration for engineering programs is four years).
            \\7.1/10
            %
            \item\textbf{ Bachelor's degree thesis }\hfill \textbf{Dec., 14, 2018}
            \\Implementation of Smart crates and IoT on the Fly Food Inventory Monitoring and Automatic Stock Renewal
            \\Relator: Prof. Chhagan Charan
            \\A novel system of intelligent food content monitoring has been developed, using intelligent sensing systems that continuously monitor perishable food levels while automatically placing orders for renewal of stock.
            This system has significant implications for warehouses and wholesalers, as it facilitates the exchange of goods using smart crates equipped with internet-of-things sensing systems linked to a web application.
            %
        \end{rSubsection2}

        %----------------------------------------------------------------------------------------
        %	CERTIFICATIONS SECTION
        %----------------------------------------------------------------------------------------

        %----------------------------------------------------------------------------------------
        %	RESEACH CONTRACTS SECTION
        %----------------------------------------------------------------------------------------

%        \begin{rSubsection2}{Research Contracts}
%
%            \item\textbf{ Research Fellow }\hfill \textbf{Oct., 2021 $\rightarrow$ Oct., 2022}
%            \\Department of Computer Science and Engineering (DISI), University of Bologna, Italy.
%            \\Project title: ``Strumenti di logica computazionale per estrazione e iniezione di conoscenza simbolica da e verso predittori subsimbolici''.
%            Goal: producing a software for the symbolic extraction/injection from/into sub-symbolic predictors.
%            \\Supervisor: Prof. \href{mailto:andrea.omicini@unibo.it}{Andrea Omicini}
%
%        \end{rSubsection2}

        %----------------------------------------------------------------------------------------
        %	OTHER CONTRACTS SECTION
        %----------------------------------------------------------------------------------------

        \begin{rSubsection2}{Other Contracts}

            \item\textbf{ Internship }\hfill \textbf{Dec, 2024}
            \\Department: Mathematics and Cybernetics, Oslo, Norway.
            \\Support the EU project THEMIS with detection of misinformation using natural language processing (NLP).

        \end{rSubsection2}

        \begin{rSubsection2}{Post Graduation Courses}

            \item\textbf{ Internship }\hfill \textbf{Dec, 2024}
            \\ Junior Cloud Specialist Course (Online)
            \\ Generation Italy
            \\ Course Info: \href{https://italy.generation.org/programs/cloud-specialist/}{Cloud Specialist Course}

        \end{rSubsection2}

    \end{rSection}
    %----------------------------------------------------------------------------------------
    %	SCIENTIFIC ACTIVITIES SECTION
    %----------------------------------------------------------------------------------------

    \begin{rSection}{Scientific Activities}

    %----------------------------------------------------------------------------------------
    %	TEACHING SECTION
    %----------------------------------------------------------------------------------------
        \begin{rSubsection2}{Teaching Activity}

            \item\textbf{ Teaching for the course JavaScript }\hfill \textbf{Jul, 2022 $->$ Sep, 2022}
            \\ CodeWeekend (Online)
            \\ Delivered online JavaScript coding lessons as part of a bootcamp organized by CodeWeekend for aspiring developers.
            \\ More about CodeWeekend:\href{https://codeweekend.net/}

        \end{rSubsection2}

        %----------------------------------------------------------------------------------------
        %	EXTRA-INSTITUTIONAL TEACHING SECTION
        %----------------------------------------------------------------------------------------

        %----------------------------------------------------------------------------------------
        %	INTERNATIONAL EXPERIENCE SECTION
        %----------------------------------------------------------------------------------------
        \begin{rSubsection2}{International Experience}

            \item\textbf{ Visiting Researcher Student (Master's Thesis)}\hfill \textbf{Aug, 2023 $\rightarrow$ Jul, 2024}
            \\Department of Informatics, University of Oslo, Norway
            \\Identification of Workflows and Files contributing to TD through Commit-Level Analysis across the GitHub Action
            \\Supervisor: Prof. \href{mailto:antonima@ifi.uio.no}{Antonio Martini}

        \end{rSubsection2}

    \end{rSection}
    %----------------------------------------------------------------------------------------
    %	OTHER ACTIVITIES AND SKILLS SECTION
    %----------------------------------------------------------------------------------------

    \begin{rSection}{Other Activities and Skills}

        %----------------------------------------------------------------------------------------
        %	FACULTY ACTIVITY SECTION
        %----------------------------------------------------------------------------------------

        %----------------------------------------------------------------------------------------
        %	SOFTWARE DEVELOPMENT
        %----------------------------------------------------------------------------------------

        %----------------------------------------------------------------------------------------
        %	TECHNICAL STRENGTHS SECTION
        %----------------------------------------------------------------------------------------
        % \newpage
        \begin{rNoListSubsection}{Technical Strengths}{}{}{}
            \begin{tabular}{ @{} >{\bfseries}l @{\hspace{6ex}} l }
                Programming Paradigms	& imperative, object oriented, functional, logic\\
                Software configuration 	& Windows and Linux installation and configuration\\
                Programming Languages 	& Python, Java, C, C++, Prolog, JavaScript\\
                Data Analysis Tools		& Pandas, NumPy, Scikitlearn, Tensorflow, Pytorch\\
                Networking 				& Socket (TCP \& UDP), HTTP, RESTful API \\
                Databases 				& SQL, MySQL, MongoDB \\
                Development tools 		& Git, Docker, Continuous Integration \\
                Markup languages 		& Markdown, \LaTeX, HTML\\
                IDEs 					& PyCharm, Visual Studio, Eclipse
            \end{tabular}
        \end{rNoListSubsection}

        %----------------------------------------------------------------------------------------
        %	LANGUAGE SECTION
        %----------------------------------------------------------------------------------------

        \begin{rNoListSubsection}{Language self-assessment}{}{}{}
            \begin{center}
                \begin{tabular}{|c|c|c|c|c|c|}
                    \hline
                    &\textbf{Listening}&\textbf{Reading}&\textbf{Interaction}&\textbf{Speaking}&\textbf{Writing}\\\hline
                    \textbf{Persian}&\multicolumn{5}{c}{Native language}\vline\\\hline
                    \textbf{English}&C1&C1&C1&C1&C1\\\hline
                    \textbf{Italian}&B1&B1&B1&B1&B1\\\hline
                    \textbf{Norwegian}&A1&A1&A1&A1&A1\\\hline
                \end{tabular}
            \end{center}
        \end{rNoListSubsection}

        %----------------------------------------------------------------------------------------
        %	ADDITIONAL INFO SECTION
        %----------------------------------------------------------------------------------------

        \begin{rSubsection2}{ Volunteering }

            \item\textbf{ Treffpunkt program }\hfill \textbf{Nov, 2023 $->$ Jul, 2024}
            \\ Red Cross, Oslo, Norway
            \\ I worked as a volunteer for the Red Cross Youth programme (Treffpunkt), which is an inclusion activity for youth.
             it aims to provide a platform for young people to connect and build relationships.

        \end{rSubsection2}

    \end{rSection}
    
    %----------------------------------------------------------------------------------------
    %	SIGNATURE SECTION
    %----------------------------------------------------------------------------------------
    \vspace{2em} %4em
    \begin{flushright}
        \today
        \\
        \vspace{1em}
        \Large$\mathcalligra{Aqila}$  $\mathcalligra{Farahmand}$

    \end{flushright}
    
\end{document}